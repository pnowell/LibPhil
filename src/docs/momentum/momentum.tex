\documentclass[12pt]{article}
\usepackage[utf8]{inputenc}
\usepackage{amsmath}
\title{Momentum}
\date{}
\begin{document}
First we take the general equation for computing the angular momentum of a system about a
particular origin.
\begin{eqnarray*}
L   &=& \sum \vec{r} \times \vec{p} \nonumber \\
    &=& \int^{M} \vec{r} \times \vec{v} \; dM
\end{eqnarray*}
Where $L$ is the angular momentum, $\vec{r}$ is the displacement of a particle from the origin
and $\vec{p}$ is that particle's linear momentum.  And since
\begin{eqnarray*}
\int^{M} dM
    &=& \int \rho \; dy \nonumber \\
    &=& \int \frac{M}{H} dy
\end{eqnarray*}
We can substitute $dM$ to get
\begin{equation*}
L = \frac{M}{H} \int \vec{r} \times \vec{v} \; dy
\end{equation*}
First we get the total angular momentum for the top line segment
\begin{eqnarray*}
L_{top}
    &=& \frac{M}{H} \int_{0}^{H}
        \begin{bmatrix}
            0 \\
            y
        \end{bmatrix} \times
        \begin{bmatrix}
            V + \left( \frac{H}{2} - y \right) \omega \\
            0
        \end{bmatrix} dy \\
    &=& \frac{M}{H} \int_{0}^{H}
        \left( -y \right)
        \left( V + \left( \frac{H}{2} - y \right) \omega \right) dy \\
    &=& \frac{M}{H} \int_{0}^{H} -y V - y\frac{H \omega}{2} + y^2 \omega dy \\
    &=& \frac{M}{H}
        \left[
            - V \int_{0}^{H} y dy
            - \frac{H \omega}{2} \int_{0}^{H} y dy
            + \omega \int_{0}^{H} y^2 dy
        \right] \\
    &=& \frac{M}{H}
        \left[
            - V \frac{H^2}{2} - \frac{H \omega}{2} \cdot \frac{H^2}{2} + \omega \frac{H^3}{3}
        \right] \\
    &=& \frac{M}{H}
        \left[
            \omega \left( \frac{H^3}{3} - \frac{H^3}{4} \right) - V \frac{H^2}{2}
        \right] \\
    &=& \omega \frac{M H^2}{12} - V \frac{M H}{2}
\end{eqnarray*}
Next we get the total angular momentum for the bottom line segment
\begin{eqnarray*}
L_{bottom}
    &=& \frac{M}{H} \int_{-H}^{0}
        \begin{bmatrix}
            0 \\
            y
        \end{bmatrix} \times
        \begin{bmatrix}
            - V - \left( \frac{H}{2} + y \right) \omega \\
            0
        \end{bmatrix} dy \\
    &=& \frac{M}{H} \int_{-H}^{0}
        y \left( V + \left( \frac{H}{2} + y \right) \omega \right) dy \\
    &=& \frac{M}{H} \int_{-H}^{0} y V + y\frac{H \omega}{2} + y^2 \omega dy \\
    &=& \frac{M}{H}
        \left[
            V \int_{-H}^{0} y dy
            + \frac{H \omega}{2} \int_{-H}^{0} y dy
            + \omega \int_{-H}^{0} y^2 dy
        \right] \\
    &=& \frac{M}{H}
        \left[
            - V \frac{\left( - H \right)^2}{2}
            - \frac{H \omega}{2} \cdot \frac{ \left( - H \right)^2}{2}
            - \omega \frac{ \left( - H \right)^3}{3}
        \right] \\
    &=& \frac{M}{H}
        \left[
            \omega \left( \frac{H^3}{3} - \frac{H^3}{4} \right) - V \frac{H^2}{2}
        \right] \\
    &=& \omega \frac{M H^2}{12} - V \frac{M H}{2}
\end{eqnarray*}
We can now see that both line segments have the same angular momentum, and so the only way
they could cancel each other out is if both of their angular momenta are zero.
\begin{eqnarray*}
    \omega \frac{M H^2}{12} - V \frac{M H}{2} &=& 0 \\
    \omega \frac{M H^2}{12} &=& V \frac{M H}{2} \\
    \omega &=& \frac{12 M H V}{2 M H^2} \\
    \omega &=& \frac{6 V}{H}
\end{eqnarray*}
And so if the angular velocity of each line segment satisfies this equality, the angular momentum
of each line segment is zero, and thus the total angular momentum of the system (about the origin)
is zero.  Thus, when they collide (in an inelastic collision), both line segments would stop
moving as well as stop spinning leaving the system with zero linear momentum and zero angular
momentum.
\end{document}
